\documentclass[12pt,a4paper,oneside]{article}

\usepackage[top=2.0cm, bottom=2.0cm, left=3cm, right=1.5cm, footskip=0.5cm]{geometry}

\usepackage{cmap}                               % Улучшенный поиск русских слов в полученном pdf-файле
\usepackage[T1,TS1,T2A]{fontenc}                    % Поддержка русских букв
\usepackage[utf8]{inputenc}[2014/04/30]         % Кодировка utf8
\usepackage[english, russian]{babel}[2014/03/24]% Языки: русский, английский


\usepackage{indentfirst}
\parindent=1.25cm
\linespread{1.3}

\usepackage{graphicx}
\usepackage{setspace}

\begin{document}

		
		\begin{center}
		\begin{spacing}{1.0}
			Федеральное государственное автономное образовательное учреждение\\
			высшего образования\\ 
			«Московский физико-технический институт (государственный университет)»\\
			Физтех-школа радиотехники и компьютерных технологий\\
			Кафедра инфокоммуникационных систем и сетей
		\end{spacing}
		\end{center}
		
		\begin{flushleft}
		\begin{spacing}{1.0}
			\textbf{Направление подготовки:} 03.04.01 Прикладные математика и физика\\
			\textbf{Направленность (профиль) подготовки:} Инфокоммуникационные и вычислительные системы и технологии
		\end{spacing}
		\end{flushleft}
			
		\hfill \break
		\hfill \break

        \begin{center}
        \begin{spacing}{1.0}
            \large\textbf{Разработка  модели и алгоритмов оценки эффективности применения процедуры слайсинга при совместном обслуживании мультимедийного трафика  и трафика  IoT}\\
		\normalsize{(бакалаврская работа)}
	    	\end{spacing}
        \end{center}
        
        \hfill \break
		\hfill \break
		\hfill \break
        
      	\begin{spacing}{1.0}
        \begin{flushright}
            \begin{minipage}{200pt}
            \textbf{Студент:}\\
            \normalsize{Коноплёв Максим Дмитриевич}\\
            \underline{\hspace{200pt}}\\
            \centerline{\small\textit{(подпись студента)}}
            \hfill \break
            \textbf{Научный руководитель:}\\
            \normalsize{Степанов Сергей Николаевич}\\
            \normalsize{д. т. н., профессор}\\
            \underline{\hspace{200pt}}\\
            \centerline{\small\textit{(подпись научного руководителя)}}\\
        \end{minipage}
        \end{flushright}
        \end{spacing}
        
    \linespread{1.3}    
	\hfill \break
	\hfill \break
	\hfill \break
	\hfill \break
	\hfill \break
	\hfill \break
	\hfill \break
	\begin{center} Москва 2019 \end{center}
	\thispagestyle{empty}

%==========================================================================
	
\newpage
\section{Аннотация}
Построена модель распределения ресурса для сети беспроводной связи стандарта LTE с функциональностью NB-IoT. В модели рассматривается процесс поступления и обслуживания двух типов трафика. Один поток образован камерами слежения. Поступление соответствующих запросов следует пуассоновской модели, если предполагается, что число источников нагрузки велико, или модели Энгсета, если предполагается, что число источников нагрузки мало.  Другой поток образован передачей информации разного рода датчиков. Появление запросов этого типа описывается пуассоновской моделью с групповым поступлением и возможностью ожидания для запросов, получивших отказ. Число мест ожидания и максимально возможное время, ограничивающее пребывание заявки на ожидании, ограничены.  С использованием модели сформулированы определения для основных характеристик качества обслуживания поступающих запросов. Среди них: доля потерянных запросов, средний объем ресурса, занятый на обслуживания каждого из рассмотренных типов трафика, среднее время передачи файлов, составляющих групповой трафик. Частными случаями анализируемой системы являются модели, в которых рассматривается процесс поступления и обслуживания только трафика видеокамер или только групповой трафик датчиков. Построена система уравнений статистического равновесия, связывающая значения стационарных вероятностей модели.  Разработанная модель и средства ее анализа могут быть использованы для исследования сценариев распределения ресурса между трафиком различных коммуникационных приложений, обслуживаемым с использованием технологий LTE и NB-IoT.
	
	
\end{document}
